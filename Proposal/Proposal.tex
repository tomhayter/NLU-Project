\pdfoutput=1
\documentclass[11pt]{article}

\usepackage{acl}
\usepackage{times}
\usepackage{latexsym}
\usepackage[T1]{fontenc}
\usepackage[utf8]{inputenc}
\usepackage{microtype}
\usepackage[superscript, biblabel, nomove]{cite}

\makeatletter \renewcommand{\@citess}[1]{\textsuperscript{\,[#1]}} \makeatother

\title{Playlist Generator using Emotion Recognition and Semantic Textual Similarity}

\author{
    Team member 1 \\
    Daniel King
    \And
    Team member 2 \\
    Thomas Hayter
}
\begin{document}
\maketitle


\section{Background and Motivation}

The music industry is a huge industry, with a market cap of around \$25.9 billion\cite{mccain_2023}, making any music-related applications potentially hugely profitable. For companies such as Spotify, they need to leverage technology in any way they can to try and keep users - with one of the main ways of doing this being the generation of playlists for subscribers to listen to. As such, we propose a novel system for playlist generation, based on selecting songs with similar semantic and emotional meanings to an input sentence or song. We believe this would be useful not only to music companies such as Spotify, who need functionality to keep users from their competitors, but also small artists who rely heavily on such playlist generating systems to have their music discovered and listened to by users across the world. Furthermore, for people who listen to music, it gives them a way to discover and listen to new music based on what kind of mood they are in, what kind of music they want to listen to and songs that they already know and like. We believe this is a challenging task as it combines multiple NLU tasks into one complete system, and we will have to make recommendations on a very small sample of user input data.
\section{Problem Statement}

The input will be either a song or a string of text that will be used to generate a playlist of songs (the output).

We will first train an Emotion Recognition model using a dataset of tweets\cite{gupta_2021}\cite{pandey_2022}. This will be used to classify song lyrics into emotions. We will input a dataset of song lyrics\cite{shah_2021} into this model and produce a dictionary keyed by emotions that stores all songs inhibiting that emotion.
Then, the user will input a song or a string of text, which will be parsed into the emotion recognition model to identify the emotions that the user wants to listen to. Then, a list of all the songs that have at least one matching emotion will be selected from the dictionary. After this, we will compare the Semantic Textual Similarity of the input string/song lyrics with each of the songs chosen previously, and rank them based on similarity. We will use this\cite{sentence-transformer} sentence transformer model to compare. A playlist of k-length will then be generated based on the most similar songs.

\section{Related Work}
Discuss what solutions already exist, e.g., what the state of the art is with respect to your chosen topic. Clarify your project aim (see Section II or VI of the COMP34812 Coursework Specifications document for an explanation/some examples of acceptable aims) in relation to existing related work.

\section{Datasets and Evaluation Resources}
The datasets we plan to use are a dataset of annotated tweets\cite{gupta_2021}\cite{pandey_2022} with the emotions they contain, to train the Emotion Recognition Model. These will be all English tweets, as we are only considering English songs. It will also be a large dataset as we want the emotion recognition to be as accurate as possible. The labels for this dataset will be the range of emotions we can capture, such as `joy', `sandess', `happiness' etc. This dataset will be split into training and testing data, so that we can evaluate the emotion recognition model.

We are also using a dataset of song lyrics\cite{shah_2021} to find the emotions in and generate the playlists. The songs database should not be too large, as we have to compare each song semantically whenever the system is run. However, we still want it to be large enough that there is a comprehensive range of emotions throughout the songs.

\section{Proposed Activities}
In the table below, specify the activities that you need to complete as you develop your proposed solution. Your activities should be planned assuming you have roughly 6 calendar weeks to work on your project (between the project proposal and project implementation deadlines). In the third column, indicate how much time you are allocating to an activity in terms of number of weeks. In the last column, put the initials of the team member who will lead a given activity.

\begin{table}[h]
\centering
\begin{tabular}{|l|l|c|c|}
\hline
\textbf{Activity} & \textbf{Any comments} & \textbf{Duration} & \textbf{Lead}\\
\hline
& & & \\
\hline
& & & \\
\hline
\end{tabular}
\end{table}


\section*{References}
Provide a list of references to support your statements/ideas, making sure they are properly cited in the above sections. You can also use this section to provide attribution to any datasets/resources you are planning to use.
\bibliographystyle{plain}
\bibliography{custom}


\end{document}
