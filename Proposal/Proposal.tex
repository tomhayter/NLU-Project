\pdfoutput=1
\documentclass[11pt]{article}

\usepackage{acl}
\usepackage{times}
\usepackage{latexsym}
\usepackage[T1]{fontenc}
\usepackage[utf8]{inputenc}
\usepackage{microtype}

\title{Your Catchy Project Title}

\author{
    Team member 1 \\
    Daniel King
    \And
    Team member 2 \\
    Thomas Hayter
}
\begin{document}
\maketitle


\section{Background and Motivation}

Provide some context as to why/how your topic is: challenging, novel or useful. In other words, try to convince your reader that this is an interesting topic (and that it is worth working on not just because it is a requirement for COMP34812).

\section{Problem Statement}
Describe in specific terms the problem that you wish to address in your project. Explain what is given/provided (what you consider as input), what is/are the NLU task(s) that you propose to implement, and what the output will be. 

\section{Related Work}
Discuss what solutions already exist, e.g., what the state of the art is with respect to your chosen topic. Clarify your project aim (see Section II or VI of the COMP34812 Coursework Specifications document for an explanation/some examples of acceptable aims) in relation to existing related work.

\section{Datasets and Evaluation Resources }
In this section, you can either: (1) describe any datasets you plan to use, if you have already found any that you can readily use; or (2) describe your plan for finding/gathering data (if you have not yet found any datasets relevant to your topic). When describing datasets, include the following details: language (English or otherwise), size (e.g., how many examples), source (e.g., news, Reddit, Twitter), and types of labels included. Importantly, discuss whether you (will) have the datasets/resources that will allow you to evaluate your proposed solution. 



\section{Proposed Activities}
In the table below, specify the activities that you need to complete as you develop your proposed solution. Your activities should be planned assuming you have roughly 6 calendar weeks to work on your project (between the project proposal and project implementation deadlines). In the third column, indicate how much time you are allocating to an activity in terms of number of weeks. In the last column, put the initials of the team member who will lead a given activity.

\begin{table}[h]
\centering
\begin{tabular}{|l|l|c|c|}
\hline
\textbf{Activity} & \textbf{Any comments} & \textbf{Duration} & \textbf{Lead}\\
\hline
& & & \\
\hline
& & & \\
\hline
\end{tabular}
\end{table}


\section*{References}
Provide a list of references to support your statements/ideas, making sure they are properly cited in the above sections. You can also use this section to provide attribution to any datasets/resources you are planning to use.

\end{document}
