\pdfoutput=1
\documentclass[11pt]{article}

\usepackage{acl}
\usepackage{times}
\usepackage{latexsym}
\usepackage[T1]{fontenc}
\usepackage[utf8]{inputenc}
\usepackage{microtype}

\title{Playlist Generation using Emotion Recognition and Semantic Textual Similarity}

\author{
    Daniel King \\
    \And
    Thomas Hayter \\
}
\begin{document}
\maketitle


\section{Introduction}

Similarly to the proposal, try to convince your reader that your project is interesting. Introduce and motivate your problem. Discuss the formulation of your task.

\section{Related Work}
Short reports do not necessarily need a dedicated related work section; if you are running out of space, you can at least cover relevant work briefly, for example when substantiating your motivations or conclusions.


\section{Methodology}
Describe what you have done in detail. Would somebody (outside of your group) reading your report be able to reproduce it in broad terms? Try to disentangle the conceptual description of your approach from the actual implementation. This conceptual view is important here; details like which libraries you used and how many lines of code your script has are unimportant. You can make use of formulae if you calculate something non-straightforward (but no need to write out standard formulae, e.g., the cross-entropy loss formula) and diagrams to depict your overall methodology. This section should describe all relevant steps, such as data-gathering and pre-processing, model definition, training objectives and so on.



\section{Evaluation}
Similarly, describe your experimental setup and what you are trying to demonstrate with the experiment, and your expectations. Try to formulate a hypothesis such as ``By doing X, we expect Y to happen, because … . If Y does not happen, then it means Z, because...''. Present the results of your experiments and show how they support or contradict your hypothesis in a coherent manner, making use of graphs and tables as necessary. 

\section{Discussion}
The results you obtained should be analysed, interpreted and discussed. Do they correspond to what you have expected? Are they surprising or unexpected? Try to find an explanation for what you have observed, but be careful about making statements that are too strong or too general. Are your explanations substantiated by your experimental evidence? How do your results relate to what others have done? This is another point in the report where you could link back to existing literature, especially when comparing your results with those obtained in previous work. Depending on the depth of the discussion, this can be either its own section, or can be merged with the evaluation section.

\section{Conclusion}
Summarise what you have done to address the problem, and what it potentially means. What could be some reasonable follow-on work based on what you have done? 

\section*{References}
Provide a list of references to support your statements/ideas, making sure that they are properly cited in the above sections. You can also use this section to provide attribution to any datasets/resources you have used.


\end{document}
